\documentclass[letterpaper,  %a4paper
               %boxit,
               %titlepage,   % separate title page
               %refpage      % separate references
              ]{jacow-2_3}   %jacow}
%
% CHANGE SEQUENCE OF GRAPHICS EXTENSION TO BE EMBEDDED
% ----------------------------------------------------
% test for XeTeX where the sequence is by default eps-> pdf, jpg, png, pdf, ...
%    and the JACoW template provides JACpic2v3.eps and JACpic2v3.jpg which
%    might generates errors, therefore PNG and JPG first
%
\makeatletter%
	\ifboolexpr{bool{xetex}}
	 {\renewcommand{\Gin@extensions}{.pdf,%
	                    .png,.jpg,.bmp,.pict,.tif,.psd,.mac,.sga,.tga,.gif,%
	                    .eps,.ps,%
	                    }}{}
\makeatother

% CHECK FOR XeTeX/LuaTeX BEFORE DEFINING AN INPUT ENCODING
% --------------------------------------------------------
%   utf8  is default for XeTeX/LuaTeX 
%   utf8  in LaTeX only realizes a small portion of codes
%
\ifboolexpr{bool{xetex} or bool{luatex}} % test for XeTeX/LuaTeX
 {}                                      % input encoding is utf8 by default
 {\usepackage[utf8]{inputenc}}           % switch to utf8

\usepackage[USenglish]{babel}			 

\usepackage[final]{pdfpages}
\usepackage{multirow}
\usepackage{ragged2e}

%
% if BibLaTeX is used
%
\ifboolexpr{bool{jacowbiblatex}}%
 {%
  \addbibresource{jacow-test.bib}
  \addbibresource{biblatex-examples.bib}
 }{}
\listfiles

%
% command for typesetting a \section like word
%
\newcommand\SEC[1]{\textbf{\uppercase{#1}}}

%%
%%   Lengths for the spaces in the title
%%   \setlength\titleblockstartskip{..}  %before title, default 3pt
%%   \setlength\titleblockmiddleskip{..} %between title + author, default 1em
%%   \setlength\titleblockendskip{..}    %afterauthor, default 1em

%\copyrightspace %default 1cm. arbitrary size with e.g. \copyrightspace[2cm]

% testing to fill the copyright space
%\usepackage{eso-pic}
%\AddToShipoutPictureFG*{\AtTextLowerLeft{\textcolor{red}{COPYRIGHTSPACE}}}

\begin{document}

\title{Bunch length measurements using CTR at the AWA with comparison to simulation}

\author{N. Neveu\thanks{nneveu@anl.gov}\textsuperscript{1}, 
	    L. Spentzouris, Illinois Institute of Technology, Chicago, USA \\
		A. Halavanau, Northern Illinois University, DeKalb, USA \\
	    J. G. Power, \textsuperscript{1}Argonne National Laboratory}
\maketitle

%
\begin{abstract}
In this paper we present electron bunch length measurements 
at the Argonne Wakefield Accelerator (AWA) photoinjector facility. 
The AWA accelerator has a large dynamic charge density range, 
with electron beam charge varying between 0.1 nC - 100 nC, 
and laser spot size diameter at the cathode between 0.1 mm - 18 mm. 
The bunch length measurements were taken at different charge densities 
using a metallic screen and a Martin-Puplett interferometer to perform 
autocorrelation scans of the corresponding coherent transition radiation (CTR). 
A liquid helium-cooled 4K bolometer was used to register the interferometer signal. 
The experimental results are compared with Impact-T and OPAL-T numerical simulations.
\end{abstract}


\section{AWA Facility}
The AWA accelerator has a large dynamic charge density range, 
with electron beam charge varying between 0.1 nC - 100 nC, 
and laser spot size diameter at the cathode between 0.1 mm - 18 mm. 

Include picture of linac 

\section{Simulations}
Describe OPAL/IMPACT? Show some plots of bunch length vs. Z?

\section{Measurement Technique}
Describe what CTR is here

\section{Experimental Setup}
Describe experimental setup used to take data

\section{Data Analysis}
Give link to code used to find bunch length

\section{Results}
Compare simulations to experimental measurements
\begin{figure}[!htb]
	%   \vspace*{-.5\baselineskip}
	\centering
	%\includegraphics*[width=174pt]{JACpic_mc}
	\caption{Layout of papers.}
	\label{l2ea4-f1}
	%   \vspace*{-\baselineskip}
\end{figure}

\section{Conclusion}
Any conclusions should be in a separate section directly preceding
the \SEC{Acknowledgment}, \SEC{Appendix}, or \SEC{References} sections, in that
order.

\section{Acknowledgment}

\section{References}





\begin{Itemize}
    \item  Either A4 (\SI{21.0}{cm}~$\times$~\SI{29.7}{cm}; \SI{8.27}{in}~$\times$~\SI{11.69}{in}) or US
           letter size (\SI{21.6}{cm}~$\times$~\SI{27.9}{cm}; \SI{8.5}{in}~$\times$~\SI{11.0}{in}) paper.
    \item  Single-spaced text in two columns of \SI{82.5}{mm} (\SI{3.25}{in}) with \SI{5.3}{mm}
           (\SI{0.2}{in}) separation. More recent versions of MSWord have a default spacing of 1.5 lines;
           authors must change this to 1 line.
    \item  The text located within the margins specified in Table~\ref{l2ea4-t1}.
\end{Itemize}
\begin{table}[hbt]
%   \vspace*{-.5\baselineskip}
   \centering
   \caption{Margin Specifications}
   \begin{tabular}{lcc}
       \toprule
       \textbf{Margin} & \textbf{A4 Paper}                      & \textbf{US Letter Paper} \\
       \midrule
           Top         & \SI{37}{mm} (\SI{1.46}{in})            & \SI{0.75}{in} (\SI{19}{mm})        \\ %[3pt]
          Bottom       & \SI{19}{mm} (\SI{0.75}{in})            & \SI{0.75}{in} (\SI{19}{mm})        \\ %[3pt]
           Left        & \SI{20}{mm} (\SI{0.79}{in})            & \SI{0.79}{in} (\SI{20}{mm})        \\ %[3pt]
           Right       & \SI{20}{mm} (\SI{0.79}{in})            & \SI{1.02}{in} (\SI{26}{mm})        \\
       \bottomrule
   \end{tabular}
   \label{l2ea4-t1}
%   \vspace*{-\baselineskip}
\end{table}

\subsection{Title and Author List}

To include a funding support statement, put an asterisk after the title and
the support text at the bottom of the first column on page~1 in \LaTeX, use $\backslash$\texttt{thanks}. See also the
subsection on footnotes.

The names of authors, their organizations/affiliations and
postal addresses should be grouped by affiliation and
listed in \SI{12}{pt} upper- and lowercase letters. The name of
the submitting or primary author should be first, followed
by the co-authors, alphabetically by affiliation. Where
authors have multiple affiliations, the secondary affiliation
may be indicated with a superscript, as shown in the
author listing of this paper. See \SEC{Annex~A} for further examples.

\subsection{Section Headings}
there should never be a column break between a heading and the
following paragraph.

\subsection{Subsection Headings}
Subsection  headings  should  not  be  numbered.
there should never be a column break
between a heading and the following paragraph.

\subsection{Paragraph Text}
The last line of a paragraph should not be
printed by itself at the beginning of a column nor should the first line of
a paragraph be printed by itself at the end of a column.

\subsection{Figures, Tables and Equations}
Place figures and tables as close to their place of mention as
possible. Lettering in figures and tables should be large enough to
reproduce clearly. Use of non-approved fonts in figures can lead to
problems when the files are processed. \LaTeX\ users should be sure to use
non-bitmapped versions of Computer Modern fonts in equations (Type\,1 PostScript
or OpenType fonts are required, Their use is described in the JACoW help
pages~\cite{jacow-help}).

Figure captions are placed below figures, and table
captions are placed above tables.

Figure captions are formatted as shown in Figs.~\ref{l2ea4-f1} and \ref{l2ea4-f2},
while table captions take the form of a heading,
with initial letters of principle words, capitalized, and
without a period at the end (see Tables~\ref{eq:units} and \ref{style-tab}).
Any reference to the contents of the table should be made from
the body of the paper rather than from within the table
caption itself.

 Single-line captions are centered in the column, while captions
that span more than one line should be justified.
The \LaTeX\ template uses the ‘booktabs’ package to
format tables.

When referring to a figure from within the text, the
convention is to use the abbreviated form [e.\,g., Fig.~1]
unless the reference is at the start of the sentence, in
which case “Figure” is written in full. Reference to a
table, however, is never abbreviated [e.\,g., Table~1].


If a displayed equation needs a number (i.\,e., it will be
referenced), place it it in parentheses, and flush with the
right margin of the column. The equation itself should be
indented and centered, as far as is possible:
\begin{equation}\label{eq:units}
    C_B=\frac{q^3}{3\epsilon_{0} mc}=\SI{3.54}{\micro eV/T}
\end{equation}

When referencing a numbered equation, use the word
“Equation” at the start of a sentence, and the abbreviated
form, “Eq.”, if in the text. The equation number is placed
in parentheses [e.g., Eq. (1)].

\subsection{Units}
	
Units should be written using the standard, roman font,
not the italic font
An unbreakable space should precede a unit (in \LaTeX\ use a “\verb|\,|”,
the template uses the ‘siunits’ package to format units).
Some examples are: \SI{3}{keV},
\SI{100}{kW}, \SI{7}{µm}. When a unit appears in a hyphenated,
compound adjective that precedes a noun, it takes on the
singular form, e.\,g., the 3.8-meter long undulator.

\subsection{References}
%
% References examples given here are not formatted using \cite as there
%			 are only reference numbers [1] and [2] in the template
%
All bibliographical and web references should be numbered and listed at the
end of the paper in a section called \SEC{References}. When citing a
reference in the text, place the corresponding reference number in square
brackets~[1]. The reference citations in the text should be numbered
in ascending order. Multiple citations should appear in
the same square bracket~[3, 4] and
with ranges where appropriate~[1--4, 10].

A URL may be included as part of a reference, but its
hyperlink should NOT be added. The usual practice is to
use a monospace font for the URL so as to help distinguish
it from normal text. The Word template uses the
Lucida Sans Typewriter font (size \SI{8}{pt}), while the ‘url’ 
package in \LaTeX\ uses the Latin Modern Typewriter font.

For authors to properly cite the resources used when researching
their papers is an obligation. In the interest of
promoting uniformity and complete citations, the IEEE
Editorial Style for Transactions and Journals has been
adopted~\cite{IEEE}. Please consult the appended material, \SEC{Annex~B},
for details. The onus is on authors to pay attention to
the details of the said style to ensure complete, accurate
and properly formatted references.

\subsection{Footnotes}

Footnotes on the title and author lines may be used for acknowledgments,
affiliations and e-mail addresses. A non-numeric sequence of characters (*, \#,
\dag, \ddag) should be used.
Word users---DO NOT use Word's footnote feature (\textbf{Insert}, \textbf{Footnote})
to insert footnotes, as this will create formatting problems. Instead, insert
the title or author footnotes manually in a text box at the bottom of the first column with a
line at the top of the text box to separate the footnotes from the rest of
the paper's text.  The easiest way to do this is to copy the text box from
the JACoW template and paste it into your own document.
These “pseudo footnotes” in the text box should only
appear at the bottom of the first column on the first page.

Any other footnote in the body of the paper should
use the normal numeric sequencing (i.\,e., 1, 2, 3)
and appear at the bottom of the same column in which
it is used.

\subsection{Acronyms}
Acronyms should be defined the first time they appear.

\section{checklist for electronic publication}

\begin{Itemize}
	\item  Use only Times or Times New Roman (standard, bold or italic) and Symbol
	fonts for text---\SI{10}{pt} minimum except references, which can be \SI{9}{pt} or \SI{10}{pt}.
	\item  Figures should use Times or Times New Roman (standard, bold or italic) and
	Symbol fonts when possible---\SI{6}{pt} minimum.
	\item  Check that citations to references appear in sequential order and
	that all references are cited.
	\item  Check that the PDF file prints correctly.
	\item  Check that there are no page numbers.
	\item  Check that the margins on the printed version are within \SI{\pm1}{mm}
	of the specifications.
	\item  \LaTeX\ users can check their margins by invoking the
	\texttt{boxit} option.
\end{Itemize}

\raggedend
\newpage


\section{acknowledgment}
Any acknowledgment should be in a separate section directly preceding
the \SEC{References} or \SEC{Appendix} section.


\section{appendix}
Any appendix should be in a separate section directly preceding
the \SEC{References} section. If there is no \SEC{References} section,
this should be the last section of the paper.

\iffalse  % only for "biblatex"
	\newpage
	\printbibliography

% "biblatex" is not used, go the "manual" way
\else

%\begin{thebibliography}{99}   % Use for  10-99  references
\begin{thebibliography}{9} % Use for 1-9 references

%\bibitem{accelconf-ref}
%	C. Petit-Jean-Genaz and J. Poole,
%	``JACoW, A service to the Accelerator Community,''
%	EPAC'04, Lucerne, July 2004, THZCH03,  p.~249,
%	\url{http://www.JACoW.org/e04/papers/THZCH03.PDF}

\bibitem{jacow-help}
	JACoW,
	\url{http://www.jacow.org}

\bibitem{IEEE}
	\textit{IEEE Editorial Style Manual},
	IEEE Periodicals, Piscataway,
	NJ, USA, Oct. 2014, pp. 34--52.
\end{thebibliography}
%\null  % this is a hack for correcting the wrong un-indent by package 'flushend' in versions before 2015

\fi

followed by the co-authors, alphabetically by affiliation. If the
author list for a given affiliation spans multiple lines,
please be sure to break the line in a manner that does not
split the author’s initials from the author’s last name. 
 The affiliation name and address
are also best kept together on the same (but not necessarily
separate) line, wherever possible. (See, for example,
the entry for GSI in the following). In cases where authors
have multiple affiliations, the secondary affiliation is
inserted below the author/primary affiliation listing and is
indicated with a superscript, as shown in the following. A
spacing of \SI{3}{pt} is added to before the secondary affiliation.

Footnotes on the title and author lines may be used for
acknowledgments and e-mail addresses, using a nonnumeric
sequence of characters (\textsuperscript{*}, \textsuperscript{†}, 
\textsuperscript{‡}, \textsuperscript{§}, \textsuperscript{\#}). In Word,
footnotes are best inserted manually in a text box at the
bottom of the first column with a line at the top of the text
box to separate the footnotes from the rest of the paper’s
text.

For examples of the preferred formatting of authors and
affiliations, please consult the following list of JACoW
collaboration members.

For manuscripts submitted by large collaborations with
potentially many tens of authors and where, additionally,
there may be page number limitations, a format consisting
of the principle author’s name and institute, followed by
“on behalf of the … collaboration”, is preferred.



\subsection{Referencing JACoW Proceedings}

The format for published JACoW proceedings papers
can be readily deduced from Refs. [1-3].

\subsubsection{Author Listing} Careful attention should be given to the
placing of commas and the use of ‘and’ in the author list.
In particular, for the case of six or more authors
(Ref. [3]), a comma also follows the penultimate author.
The preference for ‘\emph{et al.}’\ takes precedence when the number
of authors becomes large (e.g., $>$6).

\subsubsection{Paper Title} As is modern practice in references, the title
of the paper is written in sentence case, i.e., only the
initial letter of the first word in the title is capitalized.
Proper nouns, however, also have a capital. Capital letters
appearing in acronyms likewise remain unaltered

\subsubsection{Conference Proceedings} The proceedings title is written
in title case in italics using standard abbreviations,
such as Int. and Conf. The preposition, “in”, in normal
font, precedes the proceedings title. The location,
i.e., city, state (if USA), and country of the conference
venue, the month (three-letter abbreviation) and the year
the conference took place, is then listed. Finally, details
pertaining to the paper itself, such as the conference paper
ID and mandatory page numbers are given. The conference
paper ID is optional, and may be included in the
interest of facilitating a search through internet search
engines. The complete or abbreviated form for citations,
as shown in the following section, is recommended. The
former is more informative to readers outside the immediate
conference sphere. Both forms, however, ensure a
proper import into digital libraries and information
sources such as INSPIRE, Scopus, and Google Scholar.
To this end, the minimal form is also listed for convenience.
Although this form is not advocated, it nevertheless
remains acceptable.

\subsection{Referencing Periodicals and Other Sources}

The IEEE style is also shown for periodicals [6-11],
online sources [12], books [13, 14], internal reports [15],
theses [16], manuals or handbooks [17], patents [18] and
unpublished material [19, 20]. Examples of correctly
formatted references can be found at the JACoW website,
under ‘Formatting Citations’ which is reached through the
‘for Authors’ link.

\subsection{Alignment of References}

In the \LaTeX\ template, \verb|\bibliography{9}| is used for
when the total number of references is less than ten. This
should be changed to \verb|\bibliography{99}| if the number of
references is ten or more.



\patchcmd\thebibliography{\section*{REFERENCES\@mkboth {REFERENCES}{REFERENCES}}}{}{}{}
\section{PAPER PUBLISHED IN A CONFERENCE PROCEEDINGS}

\definecolor{jgreen}{cmyk}{0.81, 0.00, 0.97, 0.00}
\definecolor{jred}{cmyk}  {0.00, 0.99, 1.00, 0.00}
\definecolor{jgrepc}{cmyk}{0.74, 0.05, 1.00, 0.00}
\definecolor{jblue}{cmyk} {0.87, 0.54, 0.00, 0.00}
\definecolor{jvio}{cmyk}  {0.41, 0.82, 0.00, 0.00}
\definecolor{jbook}{cmyk} {0.28, 0.88, 0.79, 0.25}
\definecolor{jrept}{cmyk} {0.07, 0.70, 1.00, 0.00}
\definecolor{jmanu}{cmyk} {0.28, 0.77, 1.00, 0.23}
\definecolor{junpu}{cmyk} {0.00, 0.83, 0.65, 0.00}


\subsection{Complete Form}

%\begin{thebibliography}{99}   % Use for  10-99  references
\begin{thebibliography}{9} % Use for 1-9 references
	
	\bibitem{item:1-1}
	A.~Alpha and B.~T.~Beta, “An interesting paper,”
	in \textit{Proc. 1st Int. Particle Accelerator Conf. (IPAC’10)}, 
	Kyoto, Japan, May 2010, 
	paper MOP057, pp. 567--569.\\
	\textcolor{jgreen}{[Conference Proceedings, two authors; optional paper ID]}

	\bibitem{item:1-2}
	A.~Alpha \emph{et al.}, 
	“A fascinating paper about FELs,”
	in \emph{Proc. 35th Int. Free-Electron Laser Conf. (FEL’13)}, 
	New York, NY, USA, Aug. 2013, 
	paper WEP033, pp. 27--29.\\
	\textcolor{jgreen}{[Conference Proceedings, for six or more authors use \emph{et al.};	
		paper ID is optional]}
	
	\bibitem{item:1-3}	
	A.~Alpha, B.~T.~Beta, C.~Gamma, and D.~Delta, 
	“An overview of control systems,”
	in \emph{Proc. 13th Int. Conf. on Accelerator and Large Experimental Physics Control Systems (ICALEPCS’11)}, Grenoble, France, Oct. 2011, 
	paper TUP014, pp. 89--91.\\
	\textcolor{jgreen}{[Conference Proceedings, four authors; optional paper ID]}
\end{thebibliography}

\newpage
\subsection{Abbreviated Form}

\begin{thebibliography}{9} % Use for 1-9 references
	\bibitem{item:2-1}
	A.~Alpha and B.~T.~Beta, “An interesting paper,”
	in \textit{Proc. IPAC’10}, 
	Kyoto, Japan, May 2010, 
	paper MOP057, pp. 567--569.\\
	\textcolor{jgreen}{[Conference Proceedings, two authors; optional paper ID]}
	
	\bibitem{item:2-2}
	A.~Alpha \emph{et al.}, 
	“A fascinating paper about FELs,”
	in \emph{Proc. FEL’13}, 
	New York, NY, USA, Aug. 2013, 
	paper WEP033, pp. 27--29.\\
	\textcolor{jgreen}{[Conference Proceedings, for six or more authors use \emph{et al.};	
		paper ID is optional]}
	
	\bibitem{item:2-3}	
	A.~Alpha, B.~T.~Beta, C.~Gamma, and D.~Delta, 
	“An overview of control systems,”
	in \emph{Proc. ICALEPCS’11}, Grenoble, France, Oct. 2011, 
	paper TUP014, pp. 89--91.\\
	\textcolor{jgreen}{[Conference Proceedings, four authors; optional paper ID]}
\end{thebibliography}

\subsection{Minimal Form}

\begin{thebibliography}{9} % Use for 1-9 references

	\bibitem{item:3-1}
	A.~Alpha and B.~T.~Beta,
	in \textit{Proc. IPAC’10}, 
	pp. 567--569.\\
	\textcolor{jgreen}{[Conference Proceedings, two authors]}
	
	\bibitem{item:3-2}
	A.~Alpha \emph{et al.}, 
	in \emph{Proc. FEL’13}, 
	pp. 27--29.\\
	\textcolor{jgreen}{[Conference Proceedings, for six or more authors use \emph{et al.}]}
	
	\bibitem{item:3-3}	
	A.~Alpha, B.~T.~Beta, C.~Gamma, and D.~Delta, 
	in \emph{Proc. ICALEPCS’11}, 
	pp. 89--91.\\
	\textcolor{jgreen}{[Conference Proceedings, four authors]}
\end{thebibliography}

\section{UNPUBLISHED PAPER PRESENTED AT A PREVIOUS CONFERENCE}

\subsection{Complete Form}

\begin{thebibliography}{9} % Use for 1-9 references
\setcounter{enumi}{3}
 \bibitem{item:41}
	A.~Alpha and B.~T.~Beta, 
	“An interesting talk,”
	presented at the 5th Int. Particle Accelerator Conf. (IPAC’14), 
	Dresden, Germany, Jun. 2014, paper MOP057, unpublished.\\
	\textcolor{jred}{[Unpublished paper; conference name in normal font; paper
	ID may only be given if material supplementing the proceedings
	exists on the JACoW website, e.\,g., PDF of talk]}
\end{thebibliography}

\subsection{Abbreviated Form}

\begin{thebibliography}{9} % Use for 1-9 references
\setcounter{enumi}{3}
 \bibitem{item:42}
	A.~Alpha and B.~T.~Beta, 
	“An interesting talk,”
	presented at IPAC’14, 
	Dresden, Germany, Jun. 2014, paper MOP057, unpublished.\\
	\textcolor{jred}{[Unpublished paper; conference name in normal font; paper
	ID may only be given if material supplementing the proceedings
	exists on the JACoW website, e.\,g., PDF of talk]}
\end{thebibliography}


\section{PAPER PRESENTED AT THE CURRENT CONFERENCE}

\subsection{Complete Form}

\begin{thebibliography}{9} % Use for 1-9 references
\setcounter{enumi}{4}
 \bibitem{item:51}
	A.~Alpha and B.~T.~Beta, 
	“An interesting talk,”
	presented at the 7th Int. Particle Accelerator Conf. (IPAC’16), 
	Busan, Korea, May 2016, 
	paper MOAB01, this conference.\\
	\textcolor{jgrepc}{[Current conference; conference name in normal font; the
			           wording “this conference” is optional]}
\end{thebibliography}

\subsection{Abbreviated Form}

\begin{thebibliography}{9} % Use for 1-9 references
\setcounter{enumi}{4}
	\bibitem{item:52}
	A.~Alpha and B.~T.~Beta, 
	“An interesting talk,”
	presented at IPAC’16, 
	Busan, Korea, May 2016, 
	paper MOAB01, this conference.\\
	\textcolor{jgrepc}{[Current conference; conference name in normal font; the
		wording “this conference” is optional]}
\end{thebibliography}


\section{PAPER PUBLISHED IN, OR SUBMITTED TO, A PERIODICAL}

\begin{thebibliography}{99} % Use for 1-9 references
  \setcounter{enumi}{5}
	\bibitem{item:6}
		P.~Mercury \emph{et al.}, 
		“Title of paper published in journal,”
		\emph{Phy. Rev. Lett.}, vol. 114, no. 5, 
		p. 050511, Feb. 2014. \\
	\textcolor{jblue}{[Periodical, Phys. Rev. Lett.; 
		             issue no. and month may be omitted]}

	\bibitem{item:7}
		P.~Venus \emph{et al.}, 
		“New techniques in laser wakefield accelerators,”
		\emph{Phys. Rev. ST Accel. Beams}, vol. 18, 
		p. 120198, Dec.~2015.   \\
	\textcolor{jblue}{[Periodical, Phys. Rev. ST Accel. Beams; 
			              month may be omitted]}

	\bibitem{item:8}
		T.~Earth \emph{et al.}, 
		“Low dose irradiation impact on modern silicon detectors,”
		\emph{Nucl. Instr. Meth.}, vol. 692, pp. 256--280, 2014.
	\textcolor{jblue}{[Periodical, Nucl. Instr. Method.]}
	
	\bibitem{item:9}
		T.~Earth, L.~Moon, and A.~Belt, 
		“Temporal correlations of x-ray free electron lasers,”
		\emph{Optics Express}, vol. 20, pp. 11396--11404, 2012.
	\textcolor{jblue}{[Periodical, Optics Express]}

	\bibitem{item:10}
		J.~B.~Good, 
		“A paper accepted for publication,”
		\emph{Phys. Rev. Lett.}, to be published.
	\textcolor{jblue}{[Periodical, paper accepted for publication 
		              by Phys. Rev. Lett.]}

	\bibitem{item:11}
		G.~D.~Read, 
		“Title of paper submitted for publication,”
		submitted for publication.
	\textcolor{jblue}{[Paper submitted for publication; the name of the 
					  periodical does not appear]}
\end{thebibliography}

\section{ONLINE SOURCE}

\begin{thebibliography}{99} % Use for 1-9 references
  \setcounter{enumi}{11}
	\bibitem{item:121}
		JACoW, \url{http://www.jacow.org} \\
		\textcolor{jvio}{[online source; no hyperlink, no period at end of URL
						  unless there is a trailing “/” as shown below. A monospace
						  font, such as Lucida Sans Typewriter (size 8 pt), is used in
					      Word, while the ‘url’ package in \LaTeX\ uses the Latin Modern Typewriter font]}

  \setcounter{enumi}{11}
	\bibitem{item:122}
		JACoW, \url{http://www.jacow.org/}.  \\
	\textcolor{jvio}{[online source; no hyperlink, period after traling “/” in
					 URL allowed. A monospace font, such as Lucida Sans Typewriter 
					 (size 8 pt), is used in Word, while the ‘url’ package in \LaTeX\  
					 uses the Latin Modern Typewriter font]}
\end{thebibliography}

\section{CITATIONS TO BOOKS}

\begin{thebibliography}{99} % Use for 1-9 references
	\setcounter{enumi}{12}
	\bibitem{item:13}
		T.~Earth and L.~Moon, 
		“Title of chapter in the book,”
		in \emph{Title of Book}, R Mars, Ed. New York, NY, USA: 
		Wiley, 1994, pp.~42--48. \\
	\textcolor{jbook}{[Chapter in book]}
	
	\bibitem{item:14}
		A.~Belt, \emph{Title of Book}. Cambridge, MA, USA: 
		MIT Press, 1986. \\
	\textcolor{jbook}{[Book]}
\end{thebibliography}

\section{REPORTS AND THESES}

\begin{thebibliography}{99} % Use for 1-9 references
	\setcounter{enumi}{14}
	\bibitem{item:15}
		G. Jupiter \emph{et al.}, 
		“Title of report,” CERN, Geneva, Switzerland,
		Rep. CERN-2012-333, Oct. 2012.\\
	\textcolor{jrept}{[Report]}

	\bibitem{item:16}
		A.~Student, “Title of thesis,”
		Ph.D. thesis, Phys. Dept.,
		Karlsruher Institut für Technologie, Karlsruhe, 
		Germany, 2014.\\
	\textcolor{jrept}{[Thesis]}
\end{thebibliography}

\newpage

\section{MANUAL}

\begin{thebibliography}{99} % Use for 1-9 references
	\setcounter{enumi}{16}
	\bibitem{item:17}
		\emph{IEEE Editorial Style Manual}, 
		IEEE Periodicals, 
		Piscataway, NJ, USA, Oct. 2014, pp. 34-52; 
		\url{http://www.ieee.org/documents/style_manual.pdf} \\
	\textcolor{jmanu}{[Handbook/Manual, no hyperlink, no period after URL]}
\end{thebibliography}

\section{PATENTS}

\begin{thebibliography}{99} % Use for 1-9 references
	\setcounter{enumi}{17}
		\bibitem{item:18}
		A.~N.~Inventor, 
		“Title of patent,”
		Patent Authority and No., Jan. 20, 2016.

\end{thebibliography}

\newpage

\section{UNPUBLISHED WORK AND PRIVATE COMMUNICATION}

\begin{thebibliography}{99} % Use for 1-9 references
	\setcounter{enumi}{18}
	\bibitem{item:19}
		P.~Neptune, “Title of paper,” unpublished.\\
	\textcolor{junpu}{[Unpublished]}
	
	\bibitem{item:20}
	P.~Uranus, private communication, Jun. 2015.\\
	\textcolor{junpu}{[Private communication]}

\end{thebibliography}
\null  % this is a hack for correcting the wrong un-indent by package 'flushend' in versions before 2015

\subsection{Figure Captions}

Figure captions should be placed below the figure and
centered if on one line, but justified if spanning two or
more lines:
\begin{center}
	Figure 1: A one line figure caption is centered.
\end{center}
\begin{justify}
	Figure 2: A lengthy figure caption that spans 
	two lines is justified.
\end{justify}
Note the colon “:” after the figure number and the period
“.” at the end of the caption.

\newpage

When referring to a figure from within the text, the
convention is to use the abbreviated form, i.\,e., Fig.~1,
unless the reference to the figure is at the start of the sentence:

\subsection{Table Headings}

Table captions should be placed above the table and
centered if on one line, but justified if spanning two or
more lines:
\begin{center}
	Table 1: Table Heading
\end{center}
\begin{justify}
	Table 2: A Particularly Long Table Heading 
	Spanning Two Lines
\end{justify}

Note the colon ":" after the table number, that the initial
letters of the principle words in the table heading are
capitalized, and the absence of a period at the end of the
caption.

When referring to a table from within the text, the convention
is NOT to abbreviate, i.\,e., Table 1.

\subsection{Equations}

If a displayed equation requires a number, it should be
placed flush with the right margin of the column. Please
leave sufficient space immediately before and after the
equation,

\subsection{Units}

An unbreakable space should always precede a unit. In \LaTeX\ use 
a “\verb|\,|” or the ‘siunits’ package to format units.
Examples are:
\SI{3}{keV}, \SI{4}{GeV}, \SI{100}{kW}, \SI{7}{µm}.



\end{document}
	
