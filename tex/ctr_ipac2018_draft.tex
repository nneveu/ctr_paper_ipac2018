\documentclass[letterpaper,  %a4paper
               %boxit,
               %titlepage,   % separate title page
               %refpage      % separate references
              ]{jacow-2_3}   %jacow}
%
% CHANGE SEQUENCE OF GRAPHICS EXTENSION TO BE EMBEDDED
% ----------------------------------------------------
% test for XeTeX where the sequence is by default eps-> pdf, jpg, png, pdf, ...
%    and the JACoW template provides JACpic2v3.eps and JACpic2v3.jpg which
%    might generates errors, therefore PNG and JPG first
%
\makeatletter%
	\ifboolexpr{bool{xetex}}
	 {\renewcommand{\Gin@extensions}{.pdf,%
	                    .png,.jpg,.bmp,.pict,.tif,.psd,.mac,.sga,.tga,.gif,%
	                    .eps,.ps,%
	                    }}{}
\makeatother

% CHECK FOR XeTeX/LuaTeX BEFORE DEFINING AN INPUT ENCODING
% --------------------------------------------------------
%   utf8  is default for XeTeX/LuaTeX 
%   utf8  in LaTeX only realizes a small portion of codes
%
\ifboolexpr{bool{xetex} or bool{luatex}} % test for XeTeX/LuaTeX
 {}                                      % input encoding is utf8 by default
 {\usepackage[utf8]{inputenc}}           % switch to utf8

\usepackage[USenglish]{babel}			 

\usepackage[final]{pdfpages}
\usepackage{multirow}
\usepackage{ragged2e}
\usepackage{tikz}
\usetikzlibrary{shapes,arrows,backgrounds}
\usetikzlibrary{mindmap,trees}
\usetikzlibrary{decorations.pathreplacing}
\usetikzlibrary{plotmarks}
%
% if BibLaTeX is used
%
\ifboolexpr{bool{jacowbiblatex}}%
 {%
  \addbibresource{jacow-test.bib}
  \addbibresource{biblatex-examples.bib}
 }{}
\listfiles

%
% command for typesetting a \section like word
%
\newcommand\SEC[1]{\textbf{\uppercase{#1}}}

%%
%%   Lengths for the spaces in the title
%%   \setlength\titleblockstartskip{..}  %before title, default 3pt
%%   \setlength\titleblockmiddleskip{..} %between title + author, default 1em
%%   \setlength\titleblockendskip{..}    %afterauthor, default 1em

%\copyrightspace %default 1cm. arbitrary size with e.g. \copyrightspace[2cm]

% testing to fill the copyright space
%\usepackage{eso-pic}
%\AddToShipoutPictureFG*{\AtTextLowerLeft{\textcolor{red}{COPYRIGHTSPACE}}}

\begin{document}

\title{Bunch length measurements using CTR at the AWA with comparison to simulation}

\author{N. Neveu\textsuperscript{1}\thanks{nneveu@anl.gov}, 
	    L. Spentzouris, Illinois Institute of Technology, Chicago, IL, USA \\
		A. Halavanau, P. Piot\textsuperscript{2}, Northern Illinois University, DeKalb, IL, USA \\
	    J. G. Power, E. Wisniewski, C. Whiteford, \textsuperscript{1}Argonne National  Laboratory, Lemont, IL, USA \\
        S. Antipov, Euclid Techlabs LLC, Solon, OH, USA \\
	    \textsuperscript{2} also at Fermilab, Batavia, IL, USA}
\maketitle

%
\begin{abstract}
In this paper we present electron bunch length measurements 
at the Argonne Wakefield Accelerator (AWA) photoinjector facility. 
The AWA accelerator has a large dynamic charge density range, 
with electron beam charge varying between 0.1 nC - 100 nC, 
and laser spot size diameter at the cathode between 0.1 mm - 18 mm. 
The bunch length measurements were taken at different charge densities 
using a metallic screen and a Martin-Puplett interferometer to perform 
autocorrelation scans of the corresponding coherent transition radiation (CTR). 
A liquid helium-cooled 4K bolometer was used to register the interferometer signal. 
The experimental results are compared with Impact-T and OPAL-T numerical simulations.
\end{abstract}


\section{AWA Facility}
The AWA Facility houses two rf photoinjectors, both 
operating at \SI{1.3}{GHz}. The photoinjector used for 
these studies consists of a gun and solenoids followed
by six accelerating cavities, as shown in Fig.~\ref{beamline}. 
This beam line is capable of low (\SI{0.1}{nC}) and 
high charge (\SI{100}{nC}) operation.
Both beamlines utilize the 248 nm UV laser 
to generate photoelectrons with the FWHM duration ranging from 1.5 ps to 10 ps.
The bunch charge is 
routinely adjusted for depending on the requirements 
of the experiments downstream of the photoinjector.
Typical operating charges are 1, 4, 10, and \SI{40}{nC}. 
While these are the most
common operating modes, other charges have been requested 
and provided depending on the experiment.
Recent experiments include emittance exchange \cite{eex}, 
structure tests \cite{pets}, thermal emittance measurements \cite{therm}, 
and two beam acceleration \cite{tba}. 
Recently, AWA laser system was upgraded with the microlens array (MLA) setup
that yields very transversely homegeneous bunches.
The effect of the MLA generated beam on the final electron 
bunch length is not yet investigated, therefore motivating
this work.


\section{Measurement Technique}
In order to measure the bunch length, we performed an autocorrelation scan
of the CTR produced by the electron distribution \cite{Happek, WBarry}.
In brief, the CTR is transported into Martin-Puplett interferometer (MPI)
where it's split and directed into two MPI arms with a half-transparent pellicle \cite{PhysRevSTAB.9.082801}. 
The CTR beams are then combined together at the exit of the MPI with the variable path difference.
The resulting CTR intensity is registered with a liquid helium cooled IR Labs
bolometer \cite{IRlabs} as function of path difference; see Fig. \ref{facilitypics}.
The path difference is then converted into time as $\Delta \tau = 2 \Delta x$.
The resulting FWHM bunch duration is determined from the Gaussian fit of 
the interferogram; see Fig. \ref{interferogram}.
\begin{figure}
 \includegraphics[width=1.0\linewidth]{images/measurement}
 \caption{An example interferogram for Q=1 nC and laser pulse FWHM of 3 ps.}
 \label{interferogram}
\end{figure}
To alleviate charge fluctuations, we recorded 15 bolometer values for each data point.
The values were then averaged and the errorbars were deduced from the data. The data points
outside of the 3$\sigma$ bracket were considered as outliers and discarded. The resulting
interference pattern as a function of time delay in the MPI is similar to presented in Fig. \ref{interferogram}.

\section{Simulations}
Simulations of the AWA beam line shown in Fig~\ref{beamline}
were performed in the Particle In Cell code and OPAL~\cite{opal}.
The gun, accelerating cavities, and solenoids were modeled with 2D
Poisson/Superfish~\cite{fish} files. All field maps were in the T7 format.
Input parameters for the simulations are shown in Table~\ref{simparam}.
Note that on crest refers to the phase of max energy gain.
In the case of the gun, a -~$20^{\circ}$ phase is measured 
w.r.t the phase of max energy gain. 
In other words, we ran $-20^{\circ}$ off crest.

\begin{figure*}[!tbh]
	\centering
	\begin{tikzpicture}[scale=1.0, text=black]
	\input{tikz-linac.tex}
	\end{tikzpicture}	
	\caption{Beam line layout at the AWA.}
	\label{beamline}
\end{figure*}

Two scenario were simulated, a low charge case at \SI{1}{nC}, and a 
high charge case at \SI{40}{nC}. 

\begin{table}[hbt]
	%   \vspace*{-.5\baselineskip}
	\centering
	\caption{Simulation Parameters}
	\begin{tabular}{lcc}
		\toprule
		\textbf{Parameter} & \textbf{Low Charge}  & \textbf{High Charge} \\
		\midrule
		Charge       & \SI{1}{nC}        & \SI{40}{nC}    \\ %[3pt]
		Gun Gradient & \SI{65}{MV/m}     & \SI{65}{MV/m}  \\ %[3pt]
		Gun Phase    & \SI{-20}{}$^{\circ}$ & \SI{-20}{}$^{\circ}$ \\		 
		$S_1$        & \SI{500}{A}		 & \SI{500}{A}	  \\
		$S_2$		 & \SI{}{A}   	 & \SI{185}{A}		 \\
		Linac Phases & On crest          & On crest       \\
		Laser FWHM   & \SI{1.5}{ps}      & \SI{1.5}{ps}   \\ %[3pt]
		Laser Radius & \SI{2}{mm}        & \SI{9}{mm}     \\
		\bottomrule
	\end{tabular}
	\label{simparam}
	%   \vspace*{-\baselineskip}
\end{table}


\section{Experimental Setup}
The beam line layout is shown in Fig.~\ref{beamline}. 
Bunches were allowed to propagate freely to the 
CTR screen. The only focusing elements used were solenoids $S_1$ and
$S_2$. As the bunches passed the CTR screen, light was
emitted through a window located next to the screen, 
as shown in Fig.~\ref{inter}. A slit was used to prevent
background x-rays from reaching the bolometer.
After passing the slit, the light was captured 
using an interferometer outside the beam line, 
also shown in Fig~\ref{inter}. 
A remotely movable leg inside the interferometer was swept, 
and the resulting signal fed to the bolometer, 
which is shown in Fig.~\ref{bolo}. The bolometer 
was cooled with liquid helium. Period refilling of
the helium was required throughout the day in order
to keep the bolometer at \SI{4}{K}.
 
\begin{figure}\label{facilitypics}
 \includegraphics[width=0.46\linewidth]{images/bolometer}
 \includegraphics[width=0.46\linewidth]{images/interferometer}
 \caption{IR labs bolometer (left) and MPI interferometer (right) used in the experiment.}
\end{figure}

 
\begin{figure}
	\begin{tikzpicture}[every node/.style={anchor=south west,inner sep=0pt},x=1mm, y=1mm,]   
	\node (fig1) at (0,0)
	{\includegraphics[width=0.5\textwidth]{images/interferometer}};
	\node[fill=white, inner sep=2pt] (txt2) at (35,15) {Interferometer};
	\node[fill=white, inner sep=2pt, rotate=26] (txt2) at (18,19.5) {Slit};	
	\node[fill=white, inner sep=2pt, rotate=20] (txt2) at (13,27) {Window};
	\end{tikzpicture}
	\caption{Interferometer used to capture CTR light as it exited a 
			window on the beam line. }
	\label{inter}
\end{figure}

\begin{figure}
	\begin{tikzpicture}[every node/.style={anchor=south west,inner sep=0pt},x=1mm, y=1mm,]   
	\node (fig1) at (0,0)
	{\includegraphics[width=0.5\textwidth]{images/bolometer}};
	\node[fill=white, inner sep=2pt] (txt2) at (35,8) {Interferometer};	
	\node[fill=white, inner sep=2pt] (txt2) at (55,25) {Window};	
	\node[fill=white, inner sep=2pt] (txt2) at (22,35) {Bolometer};
	\end{tikzpicture}
	\caption{Bolometer. }
	\label{bolo}
\end{figure}


\section{Results}
Compare simulations to experimental measurements
\begin{figure}[!htb]
	%   \vspace*{-.5\baselineskip}
	\centering
	%\includegraphics*[width=174pt]{JACpic_mc}
	\caption{Layout of papers.}
	\label{l2ea4-f1}
	%   \vspace*{-\baselineskip}
\end{figure}

\section{Conclusion}
Any conclusions should be in a separate section directly preceding
the \SEC{Acknowledgment}, \SEC{Appendix}, or \SEC{References} sections, in that
order.

\section{acknowledgments}
We would like to thank 
Northern Illinois University (NIU) for providing the 
interferometer used in this experiment. 
We also gratefully acknowledge the computing resources
provided on Bebop, a high-performance computing cluster
operated by the LCRC at Argonne National Laboratory.
This material is based upon work supported by the 
U.S. Department of Energy, Office of Science, under 
contract number DE-AC02-06CH11357 and grant number DE-SC0015479. 
Travel to IPAC'18 supported by the Division of Physics 
of the United States National Science Foundation 
Accelerator Science Program and the Division of 
Beam Physics of the American Physical Society.


\begin{thebibliography}{9}
\bibitem{eex}
G.~Ha \emph{et al.}, “Demonstration of Current Profile 
Shaping using Double Dog-Leg Emittance Exchange Beam 
Line at Argonne Wakefield Accelerator”
in \textit{Proc. IPAC’16}, 
Busan, South Korea, May 2016, 
paper TUOBB01.\\

\bibitem{pets}
J.~Shao \emph{et al.}, “PETS....”
in \textit{Proc. IPAC’18}, 
Vancouver, Canada, May 2018, 
paper xxx.\\

\bibitem{therm}
L.~Zheng \emph{et al.}, “Measurements of Thermal Emittance 
for Cesium Telluride Photocathodes in an L-Band RF Gun”
in \textit{Proc. IPAC’17}, 
Copenhagen, Denmark, May 2017, 
paper TUPAB074.\\

\bibitem{tba}
J.~Shao \emph{et al.}, “Recent Two-Beam 
Acceleration Activities at Argonne Wakefield Accelerator Facility”
in \textit{Proc. IPAC’17}, 
Copenhagen, Denmark, May 2017, 
paper WEPVA022.\\

\bibitem{PhysRevAccelBeams.20.103404}
A.~Halavanau \textit{et al.},
%"Spatial Control of Photoemitted Electron Beams using a Micro-Lens-Array Transverse-Shaping Technique",
\newblock \emph{Phys. Rev. Accel. Beams}, 20:103404, (2017).

\bibitem{Happek}
U.~Happek, A.~J. Sievers, and E.~B. Blum,
%\newblock "Observation of coherent transition radiation",
\newblock \emph{Phys. Rev. Lett.}, 67:2962--2965, (1991).

\bibitem{WBarry}
W. Barry,
%\newblock "Measurement of subpicosecond bunch profiles using coherent transition radiation",
\newblock \emph{AIP Conference Proceedings}, 390(1):173--185, (1997).

\bibitem{PhysRevSTAB.9.082801}
D.~Mihalcea, C.~L. Bohn, U.~Happek, and P.~Piot,
%\newblock "Longitudinal electron bunch diagnostics using coherent transition radiation",
\newblock \emph{Phys. Rev. ST Accel. Beams}, 9:082801, (2006).

\bibitem{IRlabs}
IR~Labs,
\newblock \url{http://www.infraredlaboratories.com/home.html}

\bibitem{ctr}
A.~Alpha \emph{et al.}, 
“A fascinating paper about CTR,”
in \emph{Proc. FEL’13}, 
New York, NY, USA, Aug. 2013, 
paper WEP033, pp. 27--29.\\

\bibitem{opal}
A.~Adelmann \emph{et al.},
“The OPAL (Object Oriented Parallel Accelerator Library) framework,”
PSI, Zurich, Switzerland,
Rep. PSI-PR-08-02, 2008-2017.

\bibitem{fish}
\emph{Reference Manual for the POISSON/SUPERFISH Group of 
	Codes},  Los Alamos Accelerator Code Group,  
 Los Alamos, NM, USA, 
 Rep. LA-UR-87-126, Jan. 1987.
\end{thebibliography}



\null  % this is a hack for correcting the wrong un-indent by package 'flushend' in versions before 2015




\end{document}
	
